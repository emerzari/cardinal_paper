%\documentclass[11pt,letterpaper,english]{article}
\documentclass{nseJournal}
% Page design
%\usepackage[top=1in, bottom=1in, left=1in, right=1in] {geometry}
%\pagestyle{empty}

% General packages
\usepackage[T1]{fontenc}
%\usepackage{txfonts}
\usepackage{xcolor}
%\usepackage{eqnarray}
\usepackage{epsfig}
\usepackage{epstopdf}
\usepackage{graphicx}
\usepackage{mathptmx}
\usepackage{enumitem}
\usepackage{booktabs}
\usepackage{setspace}
\newcommand{\verbatimfont}[1]{\renewcommand{\verbatim@font}{\ttfamily#1}}
\usepackage{graphicx}
\usepackage{epstopdf}
\usepackage{color}
\usepackage{multirow}
\usepackage{floatrow}
\usepackage{bm}
\usepackage{amsmath}
\usepackage{hhline}
\setlength\doublerulesep{.7pt}
%\usepackage{tabularx}
\usepackage{subcaption}
%\usepackage{float}
%\usepackage{sectsty}

%\usepackage{csmacros}
\usepackage{dsfont}
%\input{csvers.tex}

% Section format
\usepackage{sectsty}
\sectionfont{\normalsize}
\subsectionfont{\normalsize}
\subsubsectionfont{\normalsize \it}
\usepackage{titlesec}
\setlength{\parskip}{\baselineskip}
\setlength{\parindent}{0pt}

\titlespacing\section{0pt}{0\parskip}{0\parskip}
\titlespacing\subsection{0pt}{0\parskip}{0\parskip}
\titlespacing\subsubsection{0pt}{0\parskip}{0\parskip}
\titleformat{\section}[block]{\normalfont\normalsize\bfseries}{\thesection .}{1em}{}
\titleformat{\subsection}[block]{\normalfont\normalsize\bfseries}{\thesubsection .}{1em}{}
\titleformat{\subsubsection}[block]{\normalfont\normalsize\bfseries}{\thesubsubsection .}{1em}{}

% Bibliography format

%\usepackage[hidelinks]{hyperref}
%\usepackage[numbers]{natbib}
%\usepackage{doi}

\renewcommand{\refname}{REFERENCES}
%\makeatletter
%\renewcommand\@biblabel[1]{#1.\em}
%\makeatother
%\setlength{\bibsep}{0pt plus 0.3ex}

% Captions
\usepackage{caption}
\captionsetup[figure]{labelfont={bf,normalsize},textfont={bf,normalsize},labelformat={default},labelsep=period,name={Figure}}

% Tables
\floatsetup[table]{capposition=top}
\captionsetup[table]{labelfont={bf,normalsize},textfont={bf,normalsize},labelformat={default},labelsep=period,name={Table}}
\renewcommand{\thetable}{\Roman{table}}
\captionsetup[sub]{font={normalsize},labelfont={bf,normalsize}}

% UDF
\definecolor{brown}{RGB}{74,68,42}
\newcommand{\tw}[1]{#1\textwidth}
\newcommand{\lw}{\linewidth}
\renewcommand{\eqref}[1]{(\ref{#1})}

\raggedright

\begin{document}
\vspace*{-0.45in}
\begin{center}
{\Large\centering\bf CARDINAL: A LOWER LENGTH-SCALE MULTIPHYSICS SIMULATOR FOR PEBBLE BED REACTORS}

\vspace{3pt}

{\bf \large Elia Merzari} \\
\large Pennsylvania State University \\
\large 228 Hallowell, University Park, PA, USA \\
{\color{brown} ebm5351@psu.edu} \\

\vspace{0.25in}

%%NOTE: please adjust as needed - requires to list UIUC
{\bf \large Haomin Yuan, Misun Min, Dillon Shaver,} \\
{\bf \large Ronald Rahaman, Patrick Shriwise, Paul Romano, } \\
{\bf \large Alberto Talamo, Yu-Hsiang Lan, } \\
\large Argonne National Laboratory \\
\vspace{0.25in}
{\bf \large Derek Gaston, Richard Martineau,} \\
\large Idaho National Laboratory \\
\vspace{0.25in}
{\bf \large Paul Fischer,} \\
\large University of Illinois, Urbana-Champaign\\
\vspace{0.25in}
{\bf \large  Yassin Hassan } \\
\large Texas A\&M University \\

\end{center}



\normalsize

\section*{ABSTRACT}

This article demonstrates a multiphysics solver for pebble-bed reactors, in particular, for Berkeley's PB-FHR (fluoride-salt-cooled high-temperature reactor) Mark-I design). The FHR is a class of advanced nuclear reactors that combines the robust coated particle fuel form from high-temperature gas-cooled reactors; the direct reactor auxiliary cooling system passive decay removal of liquid metal fast reactors; and the transparent, high volumetric heat capacitance liquid-fluoride salt working fluids (e.g., FLiBe) from molten salt reactors. This fuel and coolant combination enables FHRs to operate in a high-temperature, low-pressure design space that has beneficial safety and economic implications. The PB-FHR reactor relies on a pebble bed approach; and pebble bed reactors are, in a sense, the poster child for multiscale analysis.

Relying heavily on the MultiApp capability of Multiphysics Object-Oriented Simulation Environment (MOOSE),
we have developed Cardinal, a new platform for lower-length-scale simulation of pebble-bed cores. The lower-length-scale simulator comprises three physics: neutronics (OpenMC), thermal-fluids (Nek5000/NekRS), and fuel performance (BISON).   Cardinal tightly couples all three physics and leverages advances in MOOSE, such as the MultiApp system and the concept of MOOSE-wrapped apps. Moreover, Cardinal can utilize GPUs for accelerating solutions. In this manuscript, we discuss the development of Cardinal, verification, and validation, and demonstration simulations.

\begin{flushright}
{\bf KEYWORDS} \\
Pebble Bed, FHR, multiphysics
\end{flushright}

\doublespacing

\input intro
\input cardinal
\input vv
\input demo
\input concl


%\bibliographystyle{ieeetr}
\bibliographystyle{nureth18new}
%\bibliographystyle{ans}
%\bibliographystyle{plainnat}
%\bibliographystyle{abbrv}
\bibliography{references}
%\bibliographystyle{abbrv}

\begin{center}
\scriptsize
\framebox{\parbox{2.6in}{The submitted manuscript has been created by UChicago Argonne, LLC, Operator of Argonne National Laboratory
("Argonne").  Argonne, a U.S. Department of Energy Office of Science laboratory, is operated under Contract No. DE-AC02-06CH11357.  The U.S. Government retains for itself, and others acting on its behalf, a paid-up, nonexclusive, irrevocable worldwide license in said article to reproduce, prepare derivative works, distribute copies to the public, and perform publicly and display publicly, by or on behalf of the Government.}}
\normalsize
\end{center}


\end{document}
