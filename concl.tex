\section{Conclusions and Future Work}
\label{s:sum}


In this paper, we described the creation of Cardinal, which provides a platform for running coupled
OpenMC--Nek5000/NekRS--BISON simulations. We demonstrated Cardinal's application to FHR demo problems and
presented a limited verification and validation case. This exercise demonstrates that the MOOSE-Wrapped App
paradigm works well. Moreover, by wrapping OpenMC and Nek5000, they can now be coupled to any other MOOSE-based app.

We demonstrated the application of Cardinal with a first-of-a-kind simulation on Summit, representing a $10\times$ pebble count increase compared with previous LES simulations. This exercise demonstrates that the MOOSE-Wrapped App paradigm works well, even on a GPU platform such as Summit. In these simulations, OpenMC and BISON run on the CPU while NekRS runs on the GPUs. The current results provide a clear pathway toward full-core simulations.

Future work on Cardinal will involve a more sophisticated thermal contact treatment between pebbles,
leveraging novel ideas using MOOSE constraints. We may also explore more advanced solution transfer
mechanisms such as functional expansion tally transfers to and from OpenMC and Nek5000. We
will also work on improving the scalability of transfers. We aim to achieve massively parallel simulations going up to the entire Summit machine for full core simulations. These simulations may involve asynchronous parallel execution between the
physics. In these simulations, a multiscale approach may also be employed to simulate a select number of fuel particles directly.

Extension to gas reactor pebble beds will also be considered. As part of this effort, a radiative heat transfer models will be added to Cardinal because this is an important component of the overall heat transfer behavior.
